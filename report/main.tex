% \PassOptionsToPackage{disable}{todonotes} % 关闭todo标签
\documentclass[pdfCover]{contestTemplate} % 需要pdf封面
% \documentclass[ebook]{contestTemplate} % 电子书版本,自己看看就好

%%%%%%%%%% 行号修改需要检查makefile(不使用请忽略)
\title{测试电子设计竞赛模板}
\author{陈伯硕}
\date{\today}

\usepackage{subfiles}
\begin{document}

% TODO 提交最终版论文时别忘了关闭(第1行)
% 或者使用makefile生成的build/release.pdf
\listoftodos

\maketitle

% Generate the Table of Contents if it's needed.
\tableofcontents
\newpage

\begin{abstract}
	\todo{300字以内的设计中文摘要}

	\keywords{关键词1 \qquad
    关键词2
  }
\end{abstract}



\section{引言}

\section{设计方案}


\section{符号说明}

	\glsxtrnewsymbol[
		description={%
			an example symbol % 对符号的描述
		},
		unit={\si{m^2}} % 单位,不显示可以不写
	]{e}{\ensuremath{\mathcal{E}}}

	% 符号表
	\printunsrtglossary[type=symbols,style=symbunitlong] % symbols

	% 带单位的符号表
	% \printunsrtglossary[type=symbols,style=symblong] % symbols without units

\section{理论分析与计算}

\section{硬件电路设计}

\section{软件设计}

\section{系统测试}

	\subsection{测试仪器及测试方法}

	\subsection{测试过程及结果分析}

\section{结论}


\section*{插入代码}
\begin{codebox}
  \Procname{$bubble\_sort$($A$: the array, $n$: the length of nums)}
  \li $a = 1$
  % \li

\end{codebox}

\section*{来自inkscape的图片}
	%\cref{fig:_test}
	\begin{figure}[H]
		\centering
		\def\svgwidth{0.6\linewidth}
		\input{figures/_test.pdf_tex}
		\caption{测试图片}
		\label{fig:_test}
	\end{figure}
	\todo{测试一个TODO}


\bibliography{reference/reference}
\nocite{*}
\end{document}
